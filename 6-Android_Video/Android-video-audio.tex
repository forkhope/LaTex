\documentclass[11pt,a4paper]{article}

\usepackage{CJKutf8}
\usepackage{indentfirst}
\usepackage{graphicx}

\begin{document}
\begin{CJK}{UTF8}{gbsn}
    \title{Android音视频传输方案}
    \author{John Lee}
    \maketitle

    \section{音视频传输需求}
    具体要求:
    \begin{quote}
        提供主机以及手持端的APK及源码。
        搭建好云端或者服务器,并提供维护指南及源代码。
    \end{quote}
    \begin{enumerate}
        \item 场景中设备分为主机和手持设备,主机配备有摄像头,视频传送方向为从主机向手持设备单向传送,命令类内容和语音可双向传送,传送方式通过主机和手持设备上安装的APK。
        \item 主机常态连接WIFI,但不是固定IP,手持设备也通过WIFI连接,但有可能和主机在局域网内,也可能处于外网。
        \item 主机与手持设备均以一个固有ID登陆,通过云端或者后台服务器建立一对一连接。
        \item 手持设备APK可以请求主机的摄像机的音视频流,在手持端进行播放,手持APP也可以通过网络向主机端发送手持端的语音。
    \end{enumerate}

    \section{技术点}
    \begin{itemize}
        \item 如何实时采集主机的音视频流,如何实时采集手持设备的音频流?
        \item 如何处理采集的音视频流,如是否进行编码压缩、采用哪种编码格式?
        \item 采用哪种网络协议来传输音视频流、命令类内容,如何标识网络数据中的音视频流、命令类内容?
        \item 手持设备如何解码、实时显示接收到的音视频流?
    \end{itemize}

    \section{框架流程图}
    \begin{center}
        \includegraphics[width=\textwidth]{1-diagram/framework.png}
    \end{center}

    \section{实时采集音视频流}
    在主机端,有如下方式实时采集音视频流:
    \begin{itemize}
        \item 使用\emph{Android Camera}类的setPreviewCallback()函数,指定\\
            onPreviewFrame()接口,实时获取每一帧视频流数据。
        \item 使用\emph{Adnroid AudioRecord}类采集音频。
        \item 使用\emph{Android MediaRecorder}类同步采集视频和音频流。
        \item 使用\emph{FFmpeg}或\emph{getStreamer()}等获取Camera视频。
    \end{itemize}

    \subsection{方案一:使用Camera和AudioRecord类}
    \includegraphics[width=0.9\textwidth]{1-diagram/video-one.png}
    \par
    \textbf{优点:}
    \begin{itemize}
        \item \emph{onPreviewFrame}函数实时获取摄像头原始的每一帧视频流数据,格式为\emph{YUV(NV21)}。
        \item \emph{AudioRecord}实时获取未经压缩的音频流,格式为\emph{PCM}。
    \end{itemize}

    \textbf{缺点:}
    \begin{itemize}
        \item 不能同步采集到音频。采集视频和采集音频是分开的两个步骤。
        \item \emph{Camera}类不提供视频编码方案,需要添加代码来对采集到的\textbf{YUV}数据进行编码。
        \item \emph{AudioRecord}不提供音频编码方案,需要自行移植编码库,开源的编解码库有\emph{ilbc}、\emph{speex}等。
        \item 如果用\textbf{socket}传输,需要添加代码来发送视频流数据。
    \end{itemize}

    \subsection{方案二:使用MediaRecorder类}
    \includegraphics[width=0.9\textwidth]{1-diagram/video-two.png}
    \par
    \textbf{优点:}
    \begin{itemize}
        \item 可以指定视频编码格式,如\emph{H.263}、\emph{H.264}、\emph{MPEG-4}等,不需要额外添加代码来进行视频编码。
        \item 可以调用\emph{setAudioSource()}函数来设置音频来源,同步采集视频和音频。
        \item 可以调用\emph{setOutputFile()}函数直接绑定\textbf{socket}流,由\emph{MediaRecorder}类内部自行在\textbf{socket}上传输视频流。
    \end{itemize}

    \textbf{缺点:}
    \begin{itemize}
        \item 通过\textbf{socket}传输时,由于没有关联本地文件,无法对视频文件回写,缺少部分视频格式头信息,客户端将\textbf{socket}发送过来的数据保存成文件时,这个文件无法正常播放,需要额外解码。
        \item 客户端需要缓冲压缩后的音视频流并解码才能播放,存在一定的延时。
    \end{itemize}

    \subsection{方案三:使用FFmpeg库}
    \includegraphics[width=0.9\textwidth]{1-diagram/video-three.png}
    \par
    \textbf{FFmpeg}是一个开源的视频和音频流方案,提供了音视频采集、音视频录制,编码转换和流化音视频的功能。\par

    \textbf{优点:}
    \begin{itemize}
        \item 使用标准流媒体协议,不用主机端自行封装协议,也省去了视频编码的操作。
        \item 使用成熟的开源解决方案,功能强大、稳定、可靠。
        \item \emph{FFmpeg}可以采集音频和视频,避免额外代码来采集音频。
    \end{itemize}

    \textbf{缺点:}
    \begin{itemize}
        \item 需要自行移植\emph{FFmpeg}库到\textbf{Android}上。
    \end{itemize}

    \section{音视频流的编解码格式}
    在Android上进行摄像头预览时,可以设置预览的图片格式,如默认的\emph{NV21}、\emph{YV12}等,具体摄像头所支持的格式列表可以调用\\ \emph{getSupportedPreviewFormats()}函数来获取。\par
    Android强烈建议设置为\emph{NV21}或者\emph{YU21}格式,因为这两个格式在所有摄像头设备上都被支持。这两种设置都对应\textbf{YUV}格式。\par
    使用\emph{Android AudioRecord}类来采集音频时,获取到的是未经压缩的音频流。\par
    在网络上传输音视频流时,可以选择如下几种方案:
    \begin{itemize}
        \item 不编码。
        \item 对原始\emph{YUV}视频流进行压缩。
        \item 对采集到的音频流进行压缩。
    \end{itemize}

    \subsection{方案一:不编码}
    这个方案直接传输直接传输原始\emph{YUV}视频流和未经压缩的音频流。\par
    \textbf{优点:}
    \begin{itemize}
        \item 省去编码解码的过程,\emph{Android AudioTrack}类能够播放实时音频流,一定程度上能够简化代码。
        \item 如果能够直接播放原始音视频流,则可以实时播放,不需要缓冲播放。
    \end{itemize}

    \textbf{缺点:}
    \begin{itemize}
        \item 音视频流没有经过压缩,数据量大,传输速率低。
        \item \textbf{Android}上层不能直接显示\textbf{YUV}数据,客户端需要额外对\textbf{YUV}数据进行处理才能显示。
    \end{itemize}

    \subsection{方案二:压缩YUV视频帧}
    可以采用两种方式来压缩采集到的YUV视频帧:
    \begin{itemize}
        \item 压缩为静态的\textbf{JPEG}格式。
        \item 使用视频压缩格式。
    \end{itemize}

    \subsubsection{压缩为JPEG格式}
    这个方案将原始\emph{YUV}视频帧压缩为\textbf{JPEG}格式,并传输该\textbf{JPEG}数据。\par
    \textbf{优点:}
    \begin{itemize}
        \item \textbf{Android}原生支持,编码难度低。
    \end{itemize}

    \textbf{缺点:}
    \begin{itemize}
        \item 压缩后的\textbf{JPEG}数据还是偏大,传输速率低。
        \item \textbf{JPEG}格式为静态图片,不能很好地显示动态物体。
        \item 如果用\textbf{socket}传输,不方便识别\textbf{JPEG}图片结尾,难以选择显示\textbf{JPEG}图片的时机。一个简单的做法是主机端发送完一张图片后,就断开\textbf{socket}连接,客户端读取到\textbf{EOF}后,就显示获取到的图片。
    \end{itemize}

    \subsubsection{使用视频压缩格式}
    Android层可以使用\emph{MediaRecorder}类来采集并压缩视频流。原生支持的视频格式有:\emph{H.263}、\emph{H.264}、\emph{MPEG-4}。如果要支持更多的格式,需要移植相应的编解码库。\par
    一般来说,可以比较下面几项来判断不同视频格式之间的优劣:
    \begin{description}
        \item[视频码率] 指的是数据传输时单位时间内传送的数据位数,单位一般是kbps,即千位每秒。通俗的说就是取样率,单位时间内取样率越大,精度越高,质量越好,同时文件体积也越大。同等图像质量下,码率越低,得到的文件体积就越小。
        \item[压缩比] 压缩前和压缩后的文件所占的磁盘空间比值,称为压缩比。压缩比越高,压缩后的文件越小,占用的存储空间也越小。
        \item[图像质量] 主要包含两个方面:图像的逼真度和图像的可懂度。简单的说,压缩后的图像越清晰,图像质量就越好。
        \item[网络带宽要求] 网络带宽是指一个固定时间内,能通过的最大位数据。同样大小的原始文件,压缩后的文件越小,占用的网络带宽就越小,具有更好的适用性。
        \item[算法复杂度] 指的是该视频格式的实现复杂度。算法越复杂,就越难实现,就有更大可能带来编码错误,系统运行要求也高,比如需要更多的处理器和更多的内存。
        \item[兼容性] 同一种视频格式,不同编解码的实现是否保持兼容。显然,兼容性越差,不同产品之间的互通性就越差,可能出现某个编码器编出的数据,只能通过它的解码器来解码的情况。
    \end{description}
    \par
    目前没有找到权威、详实的数据来对比\emph{MPEG-4}、\emph{H.263}、\emph{H.264}之间的差异,只能大概列举如下:\\
    \begin{tabular}{|c|c|c|c|c|c|c|}
        \hline
        & 码率 & 压缩比 & 图像质量 & 带宽要求 & 算法复杂度 & 兼容性 \\ \hline
        H.263 & 变 & 高 & 好 & 可变 & 较高 & 互通良好 \\ \hline
        H.264 & 最低 & 比263高 & 比263好 & 可变 & 最复杂 & 互通性差 \\ \hline
        MPEG-4 & 低 & 高 & 好 & 低 & 最低 & 好 \\ \hline
    \end{tabular}
    \par
    另外,\textbf{H.263}只能传输视频,不能传输音频。而\textbf{H.264}和\textbf{MPEG-4}即能传输视频,也能传输音频。
    \subsection{方案三:压缩音频}
    在\textbf{Android}中,\emph{AudioRecord}获取到的是\textbf{PCM}格式的音频流,如果要压缩该音频流,需要移植音频编码库。\emph{MediaRecorder}能够采集并压缩音频流,原生支持的音频格式有:\emph{AMR}、\emph{AAC}。
    \begin{description}
        \item[AMR] 全称为\textbf{A}daptive \textbf{M}ulti-\textbf{R}ate (\emph{AMR-NB})和\textbf{A}daptive \textbf{M}ulti-\textbf{R}ate Wideband (\emph{AMR-WB}),该音频格式主要用在移动设备上,压缩比大,质量也比较差,但用在人声、通话上,效果还不错。
        \item[AAC] 全称为\textbf{A}dvanced \textbf{A}udio \textbf{C}oding codec(\emph{ACC-LC}),包括High\\Efficiency AAC (\emph{HE-AAC}),Enhanced Low Delay AAC (\emph{AAC-ELD})。相比于MP3,AAC格式的音质更佳,文件更小。
    \end{description}

    \section{网络传输方式}
    在网络上传输音视频流、命令类内容时,可以采用如下的方式:
    \begin{itemize}
        \item Socket传输。
        \item HTTP传输。
        \item RTP/RTSP传输。
        \item 流媒体服务方式,如\emph{live555}等。
    \end{itemize}

    \subsection{Socket传输}
    直接使用\textbf{socket}传输,不使用应用层协议。为了区分视频、音频、命令类内容,有两个思路:
    \begin{itemize}
        \item 用同一个\textbf{socket}连接,在传输的数据前面附加一个类型头,标识是所要传输的是视频、音频、还是命令类内容,相当于自定义一套协议报文。
        \item 使用三个\textbf{socket}连接,占用三个端口(\emph{port}),分别传输视频(如果同步采集音视频频的话,也传输音频)、音频、命令类内容。
    \end{itemize}

    \textbf{优点:}
    \begin{itemize}
        \item 不需要调用或移植应用层协议类包。
    \end{itemize}

    \textbf{缺点:}
    \begin{itemize}
        \item 需要自行解析、维护当前传输的是哪种类型的数据。
        \item 通过流传输,不好识别特定格式数据的末尾,对解码有所影响。
    \end{itemize}

    \subsection{HTTP传输}
    主机端实现一个小型的\textbf{HTTP}服务器,客户端向主机发送\textbf{HTTP}请求,指明所要进行的操作。\par
    \textbf{优点:}
    \begin{itemize}
        \item 通过\textbf{HTTP}请求区分所要传输的数据类型。命令类内容通过表单提交。
        \item \textbf{Android}层原生支持,方便实现。
    \end{itemize}

    \textbf{缺点:}
    \begin{itemize}
        \item 主机端的\textbf{IP}地址不固定时,需要另外的方式让客户端知道变化后的\textbf{IP}地址。例如域名、主机先用\textbf{socket}发送\textbf{IP}地址到客户端(通过云端服务器中转)等方式。
        \item \textbf{HTTP}请求的资源对应主机端的文件,要求主机先把数据存储到本地,实时性较差。
        \item \textbf{HTTP}请求由客户端发出,服务器作出响应。主机端难以主动向客户端推送命令类内容。
    \end{itemize}

    \subsubsection{RTP/RTSP传输}
    通过\emph{RTP/RTSP}协议在主机端和客户端之间双向发送请求,从而传输音频和视频。
    \begin{description}
    \item[RTP] 全称是\textbf{R}eal-time \textbf{T}ransport \textbf{P}rotocol,即实时传输协议。该协议详细说明了在互联网上传输音频和视频的标准数据包格式,通常用于流媒体系统(配合\textbf{RTSP}协议)。在\textbf{TCP/IP}协议体系中,RTP属于应用层协议,通常在\textbf{UDP}(数据包传输协议)上传输。
    \item[RTSP] 全称是\textbf{R}eal \textbf{T}ime \textbf{S}treaming \textbf{P}rotocol,即实时流传输协议,该协议定义了一对多应用程序如何有效地通过IP网络传输多媒体数据。RTSP在\textbf{TCP/IP}协议体系中属于应用层协议,但在结构上位于\textbf{RTP}和\textbf{RTCP}之上,使用\textbf{TCP}或\textbf{RTP}完成数据传输。与\textbf{HTTP}协议相比,HTTP请求由客户端发出,服务器作出响应;使用RTSP时,客户端和服务器都可以发出请求,属于双向通道。
    \end{description}

    \textbf{优点:}
    \begin{itemize}
        \item 这两个协议是针对传输多媒体数据而设计,更为专业。
    \end{itemize}

    \textbf{缺点:}
    \begin{itemize}
        \item \textbf{Android}层原生没有支持,需要移植或实现这两个协议包,工作量大。
    \end{itemize}
    \par
    另外,根据目前查找的资料,无法判断出这两个协议是否原生支持传输命令类内容、是否需要扩展。


    \subsection{流媒体服务器方式}
    搭建一个流媒体服务器,由它来负责传输流媒体数据。
    \begin{description}
        \item[流媒体] 使用流方式在网络中传送音频、视频和多媒体文件的媒体形式。
        \item[流媒体服务] 使用流式协议,如\emph{RTP/RTSP}、\emph{MMS}、\emph{RTMP}等,将视频文件传输到客户端,以供在线播放;也可从视频采集、压缩软件接收实时视频流,再以流式协议直播到客户端。
        \item[live555] 这是一个为流媒体提供解决方案的C++开源项目,实现了对标准流媒体传输协议如\emph{RTP/RTCP}、\emph{RTSP}、\emph{SIP}等的支持,同时也支持对多种音视频编码格式的音视频数据进行流化、接收和处理,例如\emph{MPEG}、\emph{H.263}、\emph{JPEG}等编码格式。
    \end{description}

    \textbf{优点:}
    \begin{itemize}
        \item 不需要在\textbf{Android}上移植或实现\textbf{RTP/RTSP}协议。
    \end{itemize}

    \textbf{缺点:}
    \begin{itemize}
        \item 需要熟悉使用\emph{live555}或者其他解决方案搭建流媒体服务器。
    \end{itemize}

    \section{客户端播放音视频流}
    在客户端,有如下方式播放获取到的音视频流:
    \begin{itemize}
        \item 使用\emph{AudioTrack}播放未经压缩的实时音频流。
        \item 使用\emph{MediaPlayer}播放\textbf{AMR}音频流。MediaPlayer只能播放文件流,需要先缓存接收到的音频再播放。
        \item 使用\emph{Android VideoView}显示视频。
        \item 使用\emph{Android ImageView}显示静态\textbf{JPEG}图片。
        \item 使用\emph{Android MediaPlayer}显示视频。
        \item 使用\emph{Android GLSurfaceView}显示\textbf{YUV}图像,需要调用\textbf{OpenGL ES}库来进行\textbf{GPU}转码,将\textbf{YUV}转成\textbf{RGB}。
    \end{itemize}
    总的来说,分为两种大的方式:播放实时流,解码后再播放。\par
    播放实时流的优缺点如下:
    \textbf{优点:}
    \begin{itemize}
        \item 高保真,实时。
    \end{itemize}

    \textbf{缺点:}
    \begin{itemize}
        \item 数据流没有经过压缩,数据量大,网络传输速率低。
        \item \textbf{Android}层显示原始\textbf{YUV}图像,需要经过转码,在软件上转,会影响CPU性能;通过GPU转,对GPU性能有要求。
        \item 原始数据保存到本地后一般无法直接播放,还是需要进行压缩编码。
    \end{itemize}
    \par

    解码后再播放的优缺点如下:\par
    \textbf{优点:}
    \begin{itemize}
        \item 传输数据流时,经过压缩编码,数据量小,方便在网络上传输。
        \item 所接收到的数据可以保存,以供后续播放。
    \end{itemize}

    \textbf{缺点:}
    \begin{itemize}
        \item 编码后的数据具有特定格式,需要接收到完整格式数据后才能解码,即需要先缓冲再解码,实时性较差。
        \item 需要移植编解码库。
    \end{itemize}

    \section{建议方案}
    综合上面的分析,有下面这样一个的建议方案。这个方案受限于自身知识面。例如我选择用\textbf{socket}传输,因为我对\textbf{socket}比较熟悉,觉得实现起来有把握,能够大概预见可能的问题。如果换一个对\textbf{RTP/RTSP}熟悉的人,他可能会选择用\textbf{RTP/RSTP}传输。\par
    \includegraphics[width=0.9\textwidth]{1-diagram/suggest.png}
    \par
    实现这个方案时,有如下需要注意的地方:
    \begin{itemize}
        \item 要绑定\textbf{socket}流到传输MediaRecorder的setOutputFile()函数中,该\textbf{socket}必须是Android的\emph{LocalSocket}类型,因为setOutputFile()函数接受FileDescriptor类型变量,LocalSocket类有一个getFileDescriptor()函数可以返回这种类型的变量,而Java的\emph{Socket}类没有这样的函数。即,无法直接通过MediaRecorder.setOutputFile()函数来绑定Java的Socket。但是LocalSocket类只能在本地使用,所以需要把写入到LocalSocket流的数据再通过Java的Socket传出去。LocalSocket会有一个发送端,一个接收端,将发送端绑定到MediaRecorder中,则MediaRecorder写入到发送端的数据可以在接收端来接收,之后再读取接收端的输入流(InputStream),就能读取到MediaRecorder采集到的数据,把这些数据写入Java socket的输出流即可。
        \item 上面提到,将MediaRecorder绑定到\textbf{socket}流时,它无法会文件进行回写,直接保存\textbf{socket}中的数据到本地文件,这个文件无法被播放。实际验证,确实如此。写入的文件中缺少了部分格式信息,不是一个有效的MPEG-4或者H.264格式文件。这需要添加代码来填充这部分格式信息。
        \item 在\textbf{Android 2.2}往上的版本中,MediaRecorder类使用setProfile()函数来设置视频和音频的输出格式、编码格式等,setOutputFormat()、setAudioEncoder()、和setVideoEncoder()函数适用于\textbf{Android 2.2 (API Level 8)}之前的版本。由于setProfile()函数调用了setOutputFormat()、setAudioEncoder()、和setVideoEncoder()等函数,如果要调用setProfile()函数,就不要调用那三个函数。在Android 4.4中,默认的输出格式是MPEG-4(对应的常量值为2),音频编码格式是AAC(对应的常量值为3),视频编码格式是H.264(对应的常量值为2)。
        \item 在X10上实际验证发现,视频编码格式为\textbf{H.264}时,音频编码格式必须是\textbf{AAC},保存下来的文件才有声音,如果音频编码格式是\textbf{AMR},保存下来的文件听不到声音。而设置视频编码格式为\textbf{MPEG-4}时,运行时报错,提示"The given video encoder 3 is not found","Construct SoftMPEG4Encoder",即不支持硬件编码\textbf{MPEG-4}格式,而是使用软件编码,长时间运行时应用会崩溃,所保存的文件无法播放。设置成\textbf{H.263}格式也会提示不支持。
    \end{itemize}

    \clearpage
\end{CJK}
\end{document}
